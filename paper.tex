\documentclass[10pt]{sig-alternate-05-2015}
%\usepackage{pdfpages}
%\usepackage{flushend}
\usepackage{lipsum}
\usepackage{subcaption}
\usepackage{xcolor}
\usepackage{hyperref} 
\usepackage{hypcap}
\usepackage{array}
\usepackage{float}
\usepackage{listingsutf8}
\usepackage{placeins}

\begin{document}
\title{Online Communication Traces in Android Memory}

\numberofauthors{8}
\author{
J{\o}rgen Ellingsen, Espen Kjellstadli Lund, Ingrid Johanne Kjesbu Larsen, Jan Petter Berg Nilsen,\\
Emil Volckmar Ry, Anders Sefjord Torbj{\o}rnsen, Magnus Omland Torgersen, Torstein Ryland Grotle
\\\{jorgeell, espenklu, ingridjl, jpnilsen, emivr, andetorb, magnuoto, torsterg\}@stud.ntnu.no
}

\maketitle
\begin{abstract}
All data are at some point present in memory, but this paper examines what traces of communication survives in memory after the application is closed down. To examine this some of the most used communication applications were installed on an Android phone, and used for communication. The phone was then rooted without rebooting and connected to a computer for memory extraction. A market top-selling cleaning tool was also downloaded and ran between data creation and extraction to simulate a user attempting to remove traces. The experiment had some limitations since the phone had to be rooted, and all the credentials for the phone. Additionally, the phone was not in daily use, and only used during the experiment. The extraction showed traces from all the used applications except Jodel. From Snapchat we found partial images and thumbnail of sent and received images, from Facebook Messenger and Whatsapp we found all communications, and after a map address search we found both the address and the current location of the phone. This showed that a lot of information is accessible in the memory after the application is closed, and this information could persist for a long time depending on the usage and memory requirements of applications run between the generation and the extraction. 
\end{abstract}

\keywords{Android; Memory; Forensics; Practical approach}

\section{Introduction}
Today the battery life of smartphones has reached the point that even with all the usage we do on our phones today, it will in most cases last the entire day. As a result of this, a smartphone can stay on for weeks, or even months at a time and old information can still be present in the memory. As most of our communication today are done with mobile applications like Facebook, GMail and Snapchat this project tries to analyse what information is stored in memory, and how long it is recoverable for the different applications. \\ \\
For data to be used by the processor in a smart phone it has to be present in memory while being presented or used. This project aims to examine what happens when the application is closed, and the data is no longer in use. For the traces to be gone the memory must not only be freed up for other applications to use, but overwritten by new data. If not, the data will survive until the device is rebooted or new applications allocate and overwrite the now free space. Additionally, some applications claim to have features like anonymity and auto-delete functionality, and applications like this should take extra care to make sure there are no fragments of the communication left in memory after the application is closed.
\\ \\
\begin{tabular}{ |p{3cm}|p{4cm}|  }
 \hline
 \multicolumn{2}{|c|}{\textbf{Definitions}} \\
 \hline
 Root, Rooting&Gaining privileges beyond what is intended by the manufacturer, effectively super-user or root (UNIX) \\
 \hline
  Memory Dump& Bit-by-bit copy of the device's random access memory.\\
 \hline
 Carving& Searching a memory dump for  files based on content, rather than on metadata.\\
 \hline
 Regular Expression & Powerful pattern searching technique.\\
 \hline
 JSON & JavaScript Object Notation is form of organizing structured data in a string form.\\
 \hline
 SSID & Service Set Identifier is a unique identifier attached to the header of packets sent over a wireless local-area network .\\
 \hline
\end{tabular}

\clearpage

\section{Background}
\label{background}
Many of todays smart phones are running Android as their operating system, and data claims it dominates the market with an 87.6\% share in the second quarter of 2016\cite{idc}. Making it essential for future mobile forensic work, for gathering information in an investigation.\\
\\
When an investigation occurs, there are several approaches to data acquisition: (1) Manual acquisition, (2) Logical acquisition, (3) Physical acquisition, (4) Brute force acquisition. The field of interest for this paper is physical acquisition of the primary storage also known as random access memory(RAM). In essence it consists of creating a bit-by-bit copy of the physical storage(memory dump). Due to RAM being volatile, it is preferred to store information about illegal activities there, to reduce the amount of traces being left behind. If data was stored in a hard drive, it would still be resident until OS tries to overwrite the same physical area. The point being that data residing on a secondary storage device, will be living longer and probably be logged more extensively.\\
\\
The Android OS is in short words a Linux-based OS, where most OS tasks are performed by open source C libraries, and Java is used for the development of Android applications. These applications are compiled to bytecode for the Java virtual machine (JVM), which is then translated for a second virtual machine which executes them. Depending on which version of Android a smart phone is running, different virtual machines are used for execution.
For Android versions 4.4 and prior, Dalvik was used. It was replaced by ART in 4.5, and is still the de-facto standard.\\
\\
Why does this matter? The virtual machines that are running Java applications, have gained several features of the Java programming language. One of these features are the memory management module, which has a built-in garbage collector(GC). It lets the user create objects without worrying about memory allocation and deallocation. Reducing the need for boilerplate code, and problems with memory leaks and such, which often are languages like C and C++ are subject to. \\
\\
The GC attempts to reclaim garbage, or make space for new objects. Depending on how it is implemented, it might remove dead data from applications. Same can be said about the trigger mechanism, however it is often based upon size limits and collision detection.\\
\\
Both ART and Dalvik use by default a method called "concurrent mark and sweep" for their GC\cite{ARTGC,DALVIKGC}. It works by traversing the heap for objects that are "reachable" or used by applications, those who are not will be regarded as free space again. Making space for potential new data to be stored within it. A figure of the process, can be seen at figure \ref{fig:mas}. Generally the heap is a region of the memory that is regarded as free memory for any process to use, however in Java it is used to store all objects created. Therefore, the GC may potentially remove one of the better sources for information, the objects. The good news are that once data is freed, no overwriting is enforced; in other words, data is not overwritten until another object takes its place\cite{DALVIKGC}. Furthermore, each application is running with its own GC and private heap, potentially delaying other applications from running their GCs.\cite{AndroidMemManagement} Creating room for better life expectancy.

\begin{figure}[h]
  \includegraphics[width=0.5 \textwidth]{figures/gc}
  \caption{Concurrent Mark and Sweep\cite{ARTGC}}
  \label{fig:mas}
\end{figure}

Because all objects created in the heap, there may be a potential trace of information of an application. Through the use of several popular online communication applications, the amount of data that can be collected through a memory dump will be explored; despite knowing that the GC might start, or extra security measures might prevent us to do so.

\subsection{Forensics Process}
A digital forensics process is used in order to conduct the most thorough analysis of the data of a device, and prevent investigators from making mistakes. It also helps ensure that every step of the process is documented. \r{A}rnes et al. describes a process with five stages: Identification, collection, examination, analysis and presentation\cite{DiFoBook}. The process in itself is an iterative process, and specifically the phases of collection, examination and analysis are done over and over again in some cases. The digital forensics process presented here is a generic process, which means it can be used on every digital device there is. By using a digital forensics process like this, the investigators are able to keep the integrity of the data and also document chain of custody which is important when a case is going to court.\\
\\
Preparation is important in the phase of identification. When looking for a mobile device, it is essential to bring equipment to charge the device in case it is turned on (live), because there may be data on the device that will disappear if is turned off, for instance data in the RAM. Devices that are turned off (dead) should not be turned on, because it can destroy data. Mobile devices that are live at the crime scene should be put in a box that isolates it from networks and blocks signals. The box will prevent the mobile device from for example remote deletion or tampering of the data on it\cite{DiFoBook}.\\
\\
The collection phase consists of retrieving data from the device (or devices) that are brought in by the investigators\cite{DiFoBook}. Brezinski and Killalea\cite{RFC3227} suggests evidence collection to be done in an order of volatility. In the case of a mobile device, the RAM should therefore be collected before the primary storage. This is done by copying data over to another system from the mobile device. While a regular copy may cause the system to overwrite existing data, write-blockers are used in order to deny the system anything other than read access, which makes a copy doable without interfering with the existing content on the device. A hash of both the copied device and the copy itself is then created, and compared so that it is no doubt that the content of the copy matches the original device\cite{DiFoBook}.This is done in order to keep the integrity of the actual system, while investigators can work on the copy of it.\\
\\
Examination is the process of finding data aqcuired in the collection phase which might be of interest to the specific case. Forensics tools can for instance be used to categorize the data found on the mobile device, and at the same time track what the investigator is doing. Some tools also offers the capability of automatically removing known files used by the operating system, which makes the process of filtering out gigabytes of date easier for the investigator of the device.\\
\\
After the phase of examination is done, analysis is done in order to see if the files contain information that might be useful in the case. This can be everything from photos, text messages or GPS localization data of the device. Time lines are often used as a help to determine what happened in what order. Log files or metadata like time stamps from files can help create a time line, which gives a better overview of what has happened. The time line created can then be expanded to contain real-life activities, which helps the investigator further understand the data. Note that when this process is done, i.e. one device has gone through the collection phase, examination phase and analysis phase, the same process starts again for a new device that might be relevant to the case. After all the devices has gone through these stages of the process, the presentation phase is started.


\section{Practical Application}
%\lipsum[1-2]
We wanted to find remaining traces in memory from communication data and application metadata. For this we used the phone Sony Xperia M2 with Android 5.1.1.

\subsection{Laboratory Environment}
The phone was factory reset to ensure a reproducible result. At this point the phone was rooted %TODO: Explain term
using the KingRoot application version 4.9.5. This process did not require restart of the phone, potentially leaving traces of memory from before rooting and making it ideal for memory forensics provided the phone can be unlocked regardless of root status. After testing the root, the phone was restarted to ensure a consistent state. As part of this process ADB USB debugging was enabled. A memory cleaning app 'Clean Master'(version 5.14.4) was installed which claimed to remove temporary files and free memory. %TODO: Write about usafe of this application in the appropriate sections

We used this tool because we wanted to check if data could survive a cleaning tool. Since we couldn't test over longer periods of time, and since the phone had very low memory\textbf{TTL memory, JP}, we used a cleaningtool to provide a more realistic environment with lot of use. 

Several applications was installed for testing. The applications installed were:
\begin{description}
\item[Facebook]
\item[Facebook Messenger]
\item[Snapchat]
\item[WhatsApp]
\item[Jodle]
\item[Google Apps] \hfill\\Preinstalled on the phone, including GMail, Maps and Chrome
\end{description}
Since this was performed after rooting of the phone, testing of remains in memory when rooting was impossible.

Github was accessed using the Chrome browser.
%Some services were accessed using the Chrome browser:
%\begin{description}
%\item[GitHub]
%\end{description}
\subsubsection*{Approach for each application}
\subsection{Process} %TODO: Bedre navn
This section outlines the steps used for preparing the services and applications for memory forensics.

\subsubsection{Creation of credentials}
We created user accounts corresponding to individual applications. A full list is below.\\
\begin{table}
\begin{tabular}{l|l|l}
Service & User Name & Password \\ 
\hline 
Facebook & mis2016forensics@gmail.com & hackmyphone01 \\ 
Snapchat & canuhackmyphone & hackmyphone07 \\ 
WhatsApp & [Phone number] & - \\ 
Google & mis2016forensics@gmail.com & hackmyphone \\ 
Github & canuHackmyphone & hackmyphone06 \\ 
\end{tabular} 
\caption{Credentials used}
\label{tbl:credentials}
\end{table}

The credentials for whatsapp has been removed from the list, however a normal phone number was used, together with a password.

After creation of the accounts in the applications, the account was used to authenticate with the applications.
% Create User accounts
% Log in to user accounts

\subsubsection{Creating searchable data}
In each of the applications, a normal session was fabricated with easily searchable data.
The types of data used for each application is specified below:
\begin{description}
\item[Facebook] \hfill\\
Information about user and friends.
\item[Facebook messenger]\hfill\\
Unique random strings and images from camera taken in application.
\item[Snapchat]\hfill\\
Images from camera taken in application.
\item[WhatsApp]\hfill\\
Unique strings and images from camera taken in application.
\item[GMail]\hfill\\
Unique random strings and images from camera taken in application.
\item[Google Chrome]\hfill\\
Credentials to GitHub. Web history.
\item[Google Maps]\hfill\\
GPS data, locations and route between locations.
\end{description}
In addition to the data listed, credentials as stored by the application is present.

\begin{figure}[h]
\centering
 \begin{subfigure}[b]{0.15\textwidth}
\includegraphics[width=\textwidth]{figures/messenger_string}
\caption{Messenger}
\end{subfigure}
 \begin{subfigure}[b]{0.2\textwidth}
\includegraphics[width=\textwidth]{figures/random_string_in_gmail}
\caption{GMail}
\end{subfigure}
\caption{Example of string used}
\end{figure}
% Send messages using the apps
% Receive messages

% Strings used
% Use of graphic images



\subsubsection{Extracting memory}
To extract the memory AMExtract was chosen. %TODO: REF
 It did not have a profile for the phone, so a profile for extraction method, size of %TODO: name of buffer
 and other properties was created and tested.
 
 The tool was then compiled using the ndk-build tool. As part of the linux kernel headers has been modified, one of the types had to be redefined using an older header file.

\subsubsection{Searching memory}
The extracted memory was searched using the credentials from table\ref{tbl:credentials} and strings which was previously entered. For this purpose the 'strings' tool was used to extract continuous regions of ASCII text and 'grep' was used to search in this text. This has the disadvantage of only finding continuous buffers using simple storage mechanisms, but is quick to execute.

\subsubsection{Carving memory}
To find other types of resources a simple file carving was attempted using 'scalpel' with matches for the following file types:
\begin{itemize}
	\item PNG
	\item JPG
	\item TIFF
\end{itemize}

\subsection{What were we looking for}
\lipsum[7]

\section{Results}
\begin{table}[H]
\centering
\begin{tabular}{|m{2cm}|m{2cm}|m{2.5cm}|}
\hline
Search phrase & Description & Findings \\
\hline
http:// & Attempting to find web related data  & Traces of communication and general use. Most specific information was facebook communication requests with names. \\
\hline
'60\textbackslash.[0-9]*,10\textbackslash.[0-9] *' &  RegEx for GPS coordinates in the area & Our specific location, as well as the location of our search target \\
\hline
password & A generic search for string "password" & Found passwords saved in browser, specified in table \ref{spesificS_table}. \\
\hline
'\{\}' & Generic JSON search & Complete Facebook information of correspondents: messages, user names and IDs, full profile pictures\\
\hline  
wifi & Generic search for wifi connections & Connection details including SSIDs \\  
\hline
fbpushnotif & String used in Facebook push notifications & List of notifications sent to the phone.\\
\hline
facebook: & String used in Facebook activity & Messages, their participants and other activity. Lots of JSON data.\\
\hline
\end{tabular}
\caption{Table of generic searches and findings}
\label{genericS_table}
\end{table}

In this section we explore our findings, what we expect to find, what we did find and other information we found. 

\subsection{Results from generic search terms}
When searching for generic information and structures, some structured information was found. This included JSON when searching for '\{' and http requests when searching for 'http://'.

A search for coordinates matching the shown regular expression was performed, which corresponds to the area the phone was located. Matches for a location in Google Maps was returned with the street address. This was from a search performed in the Google Maps application during the preparation. Searching for part of the address string revealed the address at different stages of completion.

A list of generic search terms can be found in table \ref{genericS_table}.

\subsection{Results from specific key phrases}
The passwords and usernames from table \ref{tbl:credentials} was used to detect the presence of credentials. Results are in the following table \ref{spesificS_table}. Some applications are not included due to not having any credentials, such as Jodle and Google Maps.

\begin{table}[h]
\centering
\begin{tabular}{|m{2.5cm}|m{2.5cm}|m{2.5cm}|}
\hline
App & Description & Findings \\
\hline
Facebook and Messenger & Username and password & Found username.  \\
\hline
Snapchat & Username and password & Found nothing. \\
\hline
WhatsApp & Phone number & Found phone number. \\
\hline 
Google (Gmail) & Email and password & Found email.\\
\hline
Google Chrome (GitHub password saved in browser) & Username and password. & Found username and password \\
\hline
Wifi & Wifi name and password & Found name. Found password but without context.\\  
\hline
SSID & Searched for wifi names the device had been around & Found SSID of discovered networks, but not much context.  \\  
\hline
\end{tabular}
\caption{Table of searches and findings (credentials)}
\label{spesificS_table}
\end{table}


For finding information related to Facebook, a case-insensitive search for the string Facebook was performed. This revealed information about friends of the logged in user, metadata of other users profile pictures was found from the Facebook application along with chat messages from Facebook Messenger. Information regarding whether the contact was muted, timestamp of last delivery and timestamp of last confirmed received message by the other participant. Searching for the name of the participants could identify the chat messages, since all the messages had metadata containing name of the other party.\\

For GMail and e-mail messages, the content and metadata was often found in the memory dump. As such the messages could be found by searching for the e-mail address.
%For the applications which has been provided with a unique string

\subsection{Carved images}
When carving several images were recovered. Most of these where part of stock Android OS or part of an applications assets, however when matching JPEG images, several of the pictures taken by the camera was discovered. These had no metadata. A overview of what was returned from each application can be found below:
\begin{description}
\item[Camera]\hfill\\
The full scale image was only available for a short duration after use before being stored on the device, however a small scale preview from the select image screen or camera was still in memory.
\item[WhatsApp]\hfill\\
No images could be recovered, however a reference to a file name of the image could be found in the chat conversation from memory suggesting it is stored on the device.
\item[Snapchat]\hfill\\
The full scale image was only available for a short duration after use. After some time part of the memory region containing the image was overwritten by a sequence of 0 bits before finally being removed entirely, however a small scale preview from the select image screen or camera was still in memory.
\end{description}

\section{Discussion}
%Rooted without rebooting
%Unallocated data, and partially overwritten
%Limitations
%Full access to the phone
%Phone memory size
%Timeframe
%1GB memory, 800MB dump
%Writeblock
%Kingroot - what does it do?
%Forensic process
Rooted without rebooting - In our experiment we tried two different rooting methods. One with help of a computer and a program called Odin. This method forced us to reboot the device and lose the memory. The second method was using an application called KingRoot. This application gave us root action over the phone shortly after installation. It did not need any external help to get root access.

Unallocated data, and partially overwritten

Limitations

Full access to the phone

Phone memory size

Timeframe – The project had a stri

1GB memory, 800MB dump – The mobile had 1 GB memory, but we were able to only dump 800 MB of it. This means that some of the memory are either locked or unavable for dumping due to the operation system or other locked memory information.

Writeblock – Was used to prevent overwriting important data from the RAM. If there were any possibility for overwriting the data, that could compremise the essensial information in the case and could easly be dismissed in a court of law.

The collection phase consists of retrieving data from the device (or devices) that are brought in by the investigators[?]. Brezinski and Killalea[?] suggests evidence collection to be done in an order of volatility. In the case of a mobile device, the RAM should therefore be collected before the primary storage. This is done by copying data over to another system from the mobile device. While a regular copy may cause the system to overwrite existing data, write-blockers are used in order to deny the system anything other than read access, which makes a copy doable without interfering with the existing content on the device. A hash of both the copied device and the copy itself is then created, and compared so that it is no doubt that the content of the copy matches the original device[?].This is done in order to keep the integrity of the actual system, while investigators can work on the copy of it

Kingroot – What does it do? - KingRoot is a tool used to gain root control on a mobilephone with a system exploit. KingRoot APK will not trip the Samsung KNOX and have the ability to close Sony\_RIC perfectly. Rumours says that KingRoot may interfere with the android OS, but there were no changes to be found except root access.

No real test data – Throughout the testing of the data retrieved from dumping, all the information that was gathered had been created by us.

Forensic Process

In this project the rooting of the phone and dumping the memory was done as an real forensic case would have done. First of all the phone was not turned off while it was rooted in case there was any

sensetive data in the memory as real case would have done.

\section{Conclusion}
We are able to find traces of the applications and their corresponding data. Although the data may not be used by themselves as evidence, it may be used as corroborative evidence. The lifespan of data was not strictly tested, but it is dependent on the phone's memory size and application usage. Increasing the phone's memory size is thought to expand the lifespan of allocated data in RAM.\\\\
The results confirm that data remains present despite using an cleaning tool, and applications claiming to provide anonymity and autodelete functionality. The analysis was performed with the command-line utility grep. Given a more specialized forensic tool along with more knowledge about the android memory structure, it is speculated that more data may have been extracted; thus, increasing the number of communication traces.\\\\
Further work consists of determining whether the rooting process actually is destructing potential evidence, and examining which effect an increase of RAM has on the findings that may be retrieved from analysis.

\bibliographystyle{vancouver}

\bibliography{b}  

\end{document}
