\section{Discussion}
%Rooted without rebooting
%Unallocated data, and partially overwritten
%Limitations
%Full access to the phone
%Phone memory size
%Timeframe
%1GB memory, 800MB dump
%Writeblock
%Kingroot - what does it do?
%Forensic process
To perform the memory dump, a preliminary research was initiated to discover tools compatible with the smartphone's software; where the criteria were: easy to perform, and minimal alteration of memory. Two different tools were encountered, namely Odin and KingRoot. Both ensure that the phone is rooted, which enable memory dump functionality.\\\\
It was unfortunately discovered that Odin was not up to the task, it required: an application to be installed on a computer, a USB connection, and a reboot of the phone was required. Because it requires a reboot, memory content is lost. On the other hand, KingRoot granted root access without rebooting and is therefore preferred. Nevertheless, it is not recommended to use KingRoot in a forensic case, due to the fact that the application alters the memory unpredictably. The main cause being that the application is proprietary technology, thus problematic to examine how it interacts with the system. Consequently, if this research was to be done in a more forensics setting, different methods or tools would have been reconsidered.
\subsection{Limitations}
As already mentioned, it is difficult to determine how memory is altered in the process of extracting data; due to the nature of how RAM is used in operating systems and lack of documentation. Moreover, data is dynamic and may get erased at any time. In comparison, it is far easier to block writing to hard drives with write blockers. For this reason, a limitation with the process is how integrity of data is not preserved.\\\\
Other limitations that influence our results are: a limited memory pool to extract data from, lack of SIM card for additional information, limited time to perform the study, and running all applications of interest at once.

\subsubsection{Timeframe}
Due to the limited time frame, our testing procedure has been condensed to focus on if data would survive a cleaning tool. At first, the aim was to get an idea of how long data might be resident in RAM until overwritten. As a consequence, the scope of the research has been reduced. Despite this, the results did demonstrate that Snapchat pictures only were available for a short duration after use. It does however, create room for other data to be extracted.
\subsubsection{SIM}
The smartphone in use, did not have a SIM card. As a consequence, AMExtractor was unable to extract data about contacts, phone calls, text messages, or information related to cell towers it might have been connected to. Most data had been crafted beforehand, for the sole purpose of being easily searched for.
\subsubsection{Memory}
Despite the smartphone possessing 1GB memory, only 800MB was extracted. A theory of why this is the case, is that there is no direct mapping between the virtual and physical memory. As a result, not all information was retrieved. With a modern phone, more traces would have been acquired. Generally, because they tend to have more RAM than their ancestors. For instance, the latest Google Nexus 6P has 3GB RAM, which is three times more than the one tested\cite{huawei}.
\subsubsection{Unallocated, and partially overwritten data}
Preparation for memory collection, was done by performing various actions within each application of interest. This was done sequentially, which could have resulted in data being overwritten in the RAM. The reason being that the GC freed some of the memory fore reallocation, and as more applications were launched the memory was reallocated for them. As a result, some images that were extracted were partly overwritten and nearly unidentifiable.\\\\
It is likely that the GC probably would have done less damage to data, if more RAM was available. Simply because, the GC is primarily triggered by size limits or collision detection.
\subsubsection{Forensic Process}
The forensic phase did not follow best practice. A JTAG approach, may have been a better approach to keep data integrity but was impractical for this study. Furthermore, no specialized tools were used to perform the analysis of extracted data; the command-line utility grep was used. It is therefore likely that a more suited tool, may have given us the opportunity to find more data of relevance. Also, it cannot be confirmed that the phone has been tampered with during the process. It is unlikely, but not provable.\\