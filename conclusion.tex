\section{Conclusion}
We are able to find traces of the applications and their usage. The artefacts found could be used as corroborative evidence.

Data privacy is more important than ever, as adoption of smartphones keep increasing. This project has shown that with a few steps the amount of private information can be retrieved out of the memory on a mobile phone. First of all to gain root access over the phone was tested with two different programs where one of them did not need the phone to be turned off. That is advantageous in real forensic cases due to the fact that all the memory in the RAM is still intact. With the help of a tool called AMExtractor and the Linux kernel, we were able to dump the files from the RAM onto our computer. During the forensics process two types of searches was used in order to examine the information. A generic search with the keyword HTTP lead to us finding the GPS coordinates of the previous Google maps search on the phone. The second type of search was a more specific search with the keyword Facebook. This gave us conversations done on Facebook with the timestamps of the various messages. With the previous findings we tried a cleaning tool while closing the applications used. The result of this was little to nothing and had no impact of the information that was dumped. One of the more interesting result of our forensic process was the dumped information by the application Snapchat. Shortly after using Snapchat and dump the memory on to our computer, we found a full scale image that was taken in Snapchat. After a while the memory containing the image was overwritten, nonetheless a small scale preview of the image was still in memory. While all of our findings was fascinating, the GPS coordinates and the Facebook conversations was by far the most interesting. That kind of information can be used in real forensic cases.
%TODO: Rewrite/improve conclusion. To long and should summarise better.