\section{Conclusion}
We are able to find traces of the applications and their corresponding data. Although the data may not be used by themselves as evidence, it may be used as corroborative evidence. The lifespan of data was not strictly tested, but it is dependent on the phone's memory size and application usage. Increasing the phone's memory size is thought to expand the lifespan of allocated data in RAM.\\\\
The results confirm that data remains present despite using an cleaning tool, and applications claiming to provide anonymity and autodelete functionality. The analysis was performed with the command-line utility grep. Given a more specialized forensic tool along with more knowledge about the android memory structure, it is speculated that more data may have been extracted; thus, increasing the number of communication traces.\\\\
To extract data, the phone was rooted using the application KingRoot. Could this application manipulate or corrupt data? The answer is yes. However, this is not part of the scope for this paper.\\\\
Further work consists of determining whether the rooting process actually is destructing potential evidence, and examining which effect an increase of RAM has on the findings that may be retrieved from analysis.