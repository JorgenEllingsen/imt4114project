\section{Conclusion}
Easy to extract memory – To extract the memory from the phone a tool called AMExtract was used. It did not have a profile for the mobilephone that was used in this project. Therefore one was created and compiled using a ndk-build tool. As soon as the profile was created the memory dumping was a success.

Both of generic and specific searches - During our experimentation fase we used two different searches through the memory in order to gather information. One of the searches was a generic type of search. This means that a searchword with no other specification than HTTP was used. The second type of seacrh was a more specific one. The search word facebook was used and found conversations with both the message and the timestamps of the messages.

Cleaning tool and closing applications had little effect – With the use of a cleaning tool and closing the different apps, it did not have any effect on the amount nor what type of information that was dumped. Regardless the result was the same.

Found data that was never saved (Snapchat) – After going through the data that was dumped from the mobilephone, a full scale image was found shortly after using the application Snapchat. Afterwards the memory containing the image was overwritten, nonetheless a small scale preview of the image was still in memory.
 \nocite{Androulidakis2016}