\section{Results}
In this section we explore our findings, what we expect to find, what we did find and other information we found. 

\subsection{Results from generic search terms}
When searching for generic information and structures, some structured information was found. This included JSON when searching for '\{' and http requests when searching for 'http://'.

\begin{table}[H]
\centering
\label{genericS_table}
\begin{tabular}{|m{2.5cm}|m{2cm}|m{2.5cm}|}
\hline
Search phrase & Description & Findings \\
\hline
http:// & Attempting to find web related data  & Traces of communication and general use. Most specific information was facebook communication requests with names. \\
\hline
'60\textbackslash.[0-9]*,10\textbackslash.[0-9] *' &  RegEx for GPS coordinates in the area & Our specific location, as well as the location of our search target \\
\hline
password & A generic search for string "password" & Found passwords saved in browser, specified in table \ref{spesificS_table}. \\
\hline
'\{\}' & Generic JSON search & Complete facebook information of correspondents: messages, user names and IDs, full profile pictures\\
\hline  
wifi & Generic search for wifi connections & Connection details including SSIDs \\  
\hline
fbpushnotif & String used in Facebook push notifications & List of notifications sent to the phone.\\
\hline
facebook: & String used in facebook activity & Messages, their participants and other activity. Lots of JSON data.\\
\hline
\end{tabular}
\caption{Table of generic searches and findings}
\end{table}

A search for coordinates matching the shown regular expression was performed, which corresponds to the area the phone was located. Matches for a location in Google Maps was returned with the street address. This was from a search performed in the Google Maps application during the preparation. Searching for part of the address string revealed the address at different stages of completion.


\subsection{Results from specific key phrases}
The passwords and usernames from table \ref{tbl:credentials} was used to detect the presence of credentials. Results are in the following table \ref{spesificS_table}. Some applications are not included due to not having any credentials, such as Jodle and Google Maps.

\begin{table}[h]
\centering
\label{spesificS_table}
\begin{tabular}{|m{2.5cm}|m{2.5cm}|m{2.5cm}|}
\hline
App & Description & Findings \\
\hline
Facebook and Messenger & Username and password & Found username.  \\
\hline
Snapchat & Username and password & Found nothing. \\
\hline
WhatsApp & Phone number & Found phone number. \\
\hline 
Google (Gmail) & Email and password & Found email.\\
\hline
Google Chrome (GitHub password saved in browser) & Username and password. & Found username and password \\
\hline
Wifi & Wifi name and password & Found name. Found password but without context.\\  
\hline
SSID & Searched for wifi names the device had been around & Found SSID of discovered networks, but not much context.  \\  
\hline
\end{tabular}
\caption{Table of searches and findings (credentials)}
\end{table}


For finding information related to Facebook, a case-insensitive search for the string Facebook was performed. This revealed information about friends of the logged in user, metadata of other users profile pictures was found from the Facebook application along with chat messages from Facebook Messenger. Information regarding whether the contact was muted, timestamp of last delivery and timestamp of last confirmed received message by the other participant. Searching for the name of the participants could identify the chat messages, since all the messages had metadata containing name of the other party.\\

For GMail and e-mail messages, the content and metadata was often found in the memory dump. As such the messages could be found by searching for the e-mail address.
%For the applications which has been provided with a unique string

\subsection{Carved images}
When carving several images were recovered. Most of these where part of stock Android OS or part of an applications assets, however when matching JPEG images, several of the pictures taken by the camera was discovered. These had no metadata. A overview of what was returned from each application can be found below:
\begin{description}
\item[Camera]\hfill\\
The full scale image was only available for a short duration after use before being stored on the device, however a small scale preview from the select image screen or camera was still in memory.
\item[WhatsApp]\hfill\\
No images could be recovered, however a reference to a file name of the image could be found in the chat conversation from memory suggesting it is stored on the device.
\item[Snapchat]\hfill\\
The full scale image was only available for a short duration after use. After some time part of the memory region containing the image was overwritten by a sequence of 0 bits before finally being removed entirely, however a small scale preview from the select image screen or camera was still in memory.
\end{description}