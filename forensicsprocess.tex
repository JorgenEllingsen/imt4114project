
A digital forensics process is used in order to conduct the most thorough analysis of the data of a device, and prevent investigators from making mistakes. It also helps ensuring that every step of the process is documented. Årnes et al. describes a process with five stages: Identification, collection, examination, analysis and presentation\cite{DiFoBook}. The process in itself is an iterative process, and specifically the phases of collection, examination and analysis are done over and over again in some cases. The digital forensics process presented here is a generic process, which means it can be used on every digitial device there is. By using a digital forensics process like this, the investigators are able to keep the integrity of the data and also document chain of custody which is important when a case is going to court. \\

Preparation is important in the phase of identification. When looking for a mobile device, it is essential to bring equipment to charge the device in case it is turned on (live), because there may be data on the device that will disappear if is turned off, for instance data in the RAM. Devices that are turned off (dead) should not be turned on, because it can destroy data. Mobile devices that are live at the crime scene should be put in a box that isolates it from networks and blocks signals. The box will prevent the mobile device from for example remote deletion or tampering of the data on it\cite{DiFoBook}. 

The collection phase consists of retrieving data from the device (or devices) that are brought in by the investigators\cite{DiFoBook}. Brezinski and Killalea\cite{RFC3227} suggests evidence collection to be done in an order of volatility. In the case of a mobile device, the RAM should therefore be collected before the primary storage. This is done by copying data over to another system from the mobile device. While a regular copy may cause the system to overwrite existing data, write-blockers are used in order to deny the system anything other than read access, which makes a copy doable without interfering with the existing content on the device. A hash of both the copied device and the copy itself is then created, and compared so that it is no doubt that the content of the copy matches the original device\cite{DiFoBook}.This is done in order to keep the integrity of the actual system, while investigators can work on the copy of it. 

Examination is the process of finding data aqcuired in the collection phase which might be of interest to the specific case. Forensics tools can for instance be used to categorize the data found on the mobile device, and at the same time track what the investigator is doing. Some tools also offers the capability of automatically removing known files used by the operating system, which makes the process of filtering out gigabytes of date easier for the investigator of the device.

After the phase of examination is done, analysis is done in order to see if the files contain information that might be useful in the case. This may be everything from photos, text messages or GPS localization data of the device. Timelines are often used as a help to determine what happened in what order. Log files or meta data like time stamps from files can help create a timeline, which gives a better overview of what has happened. The timeline created can then be expanded to contain real-life activities, which helps the investigator further understand the data. Note that when this process is done, i.e. one device has gone through the collection phase, examination phase and analysis phase, the same process starts again for a new device that might be relevant to the case. After all the devices has gone through these stages of the process, the presentation phase is started.

\bibitem{DiFoBook}
Årnes, A., Flaglien, A. O., Sunde, I. M., Dilijonaite, A., Hamm, J., Sandvik, J. P., Bjelland, P. C., Franke, K., Axelsson, S (2016) 'Digital Forensics', Gjøvik: NTNU Gjøvik.
\bibitem{RFC3227}
Brezinski, D., Killalea, T. (2002) 'Guidelines for Evidence Collection and Archiving' in: Request for Comments 3227 (RFC3227).
%URL: https://tools.ietf.org/html/rfc3227#section-2.1https://tools.ietf.org/html/rfc3227#section-2.1
%eller:
%D. Brezinski and T. Killalea. Rfc 3227: Guidelines for evidence collection and archiving. Internet Engineering Task Force, 2002
%?
