\section{Introduction}
Today the battery life of smartphones has reached the point that even with all the usage we do on our phones today, it will in most cases last the entire day. As a result of this, a smartphone can stay on for weeks, or even months at a time and old information can still be present in the memory. As most of our communication today are done with mobile applications like Facebook, GMail and Snapchat this project tries to analyze what information is stored in memory, and how long it is recoverable for the different applications. \\ \\
For data to be used by the processor in a smart phone it has to be present in memory while being presented or used. This project aims to examine what happens when the applcation is closed, and the data is no longer in use. For the traces to be gone the memory must not only be freed up for other applications to use, but overwritten by new data. If not, the data will survive until the device is rebooted or new applications allocate and overwrite the now free space. Additionally, some applications claim to have features like anonymity and auto-delete functionality, and applications like this should take extra care to make sure there is no fragments of the communication left in memory after the application is closed.
\\ \\
\begin{tabular}{ |p{3cm}|p{4cm}|  }
 \hline
 \multicolumn{2}{|c|}{\textbf{Definitions}} \\
 \hline
 Root, Rooting&Gaining privileges beyond whats intended by the manufacturer, effectivly super-user or root (UNIX) \\
 \hline
  Memory Dump& Bit-by-bit copy of the device's random access memory.\\
 \hline
 Carving& Searching a memory dump for  files based on content, rather than on metadata.\\
 \hline
 Regular Expression & Powerful pattern searching technique.\\
 \hline
 JSON & JavaScript Object Notation is form of organizing structured data in a string form.\\
 \hline
\end{tabular}